\begin{table}
\centering
\caption{Perete in X-LAM con isolante alta densità fibra di legno}
\begin{tabular}{lrrrrrrr}
\toprule
                                    Strati & Spessori & Densità & Calore specifico & Massa superficiale & Profondità di Penetrazione &     xi & Capacità termica & Diffusività termica x 10-7 & Effusività termcia \\
\midrule
                                Gessofibra &    0,015 &  1150,0 &           1100,0 &              17,25 &                      0,068 &  0,222 &             19,0 &                       1,66 &              515,4 \\
                                 X-LAM KLH &    0,096 &   500,0 &           1600,0 &              48,00 &                      0,067 &  1,436 &             76,8 &                       1,62 &              322,5 \\
 Isolante alta densità Naturalia Diffuterm &    0,130 &   190,0 &           2100,0 &              24,70 &                      0,054 &  2,388 &             51,9 &                       1,08 &              131,0 \\
                            Intonaco calce &    0,015 &  1800,0 &           1000,0 &              27,00 &                      0,117 &  0,128 &             27,0 &                       5,00 &             1272,8 \\
\bottomrule
\end{tabular}
\end{table}

\begin{flushleft}
\begin{align*}
\text{Massa superficiale totale} \, M_s &= \SI{116.95}{\kilo\gram\per\metre\squared}\\
\text{Sfasamento} \, \Delta\tau &= \SI{13.78}{\hour}\\
\text{Fattore di attenuazione} \, fd &= \SI{0.177}{}\\
\text{Trasmittanza termica periodica} \, Y_{12} &= \SI{0.044}{\watt\per\metre\squared\per\kelvin}\\
\text{Ammettanza termica interna} \, Y_{11} &= \SI{2.613}{\watt\per\metre\squared\per\kelvin}\\
\text{Ammettanza termica esterna} \, Y_{22} &= \SI{2.73}{\watt\per\metre\squared\per\kelvin}\\
\text{Capacità termica periodica interna} \, k_1 &= \SI{36.21}{\kilo\joule\per\metre\squared\per\kelvin}\\
\end{align*}
\end{flushleft}
