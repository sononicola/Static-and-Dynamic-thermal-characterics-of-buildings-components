\begin{table}[H]
\centering
\caption{Perete in X-LAM con isolante bassa densità lana di roccia}
\resizebox{\linewidth}{!}{%
\begin{tabular}{@{}
				l
				S[table-format=1.3]
				S[table-format=4.1]
				S[table-format=4.1]
				S[table-format=2.2]
				S[table-format=1.3]
				S[table-format=1.3]
				S[table-format=2.1]
				S[table-format=1.1]
				S[table-format=4.1]
				@{}
				}
\toprule
\multicolumn{1}{c}{\multirow{3}{*}{Strati}} & \multicolumn{1}{c}{\multirow{2}{*}{Spessori}}    & \multicolumn{1}{c}{\multirow{2}{*}{Densità}}                                            & \multicolumn{1}{c}{Calore}                                            & \multicolumn{1}{c}{Massa}                                               & \multicolumn{1}{c}{Profondità di}                      & \multicolumn{1}{c}{Rapporto}   & \multicolumn{1}{c}{Capacità}                                                  & \multicolumn{1}{c}{Diffusività}                                      & \multicolumn{1}{c}{Effusività}                                                             \\
\multicolumn{1}{c}{}                        & \multicolumn{1}{c}{}                              & \multicolumn{1}{c}{}                                                  & \multicolumn{1}{c}{specifico}                                         & \multicolumn{1}{c}{superficiale}                                        & \multicolumn{1}{c}{penetrazione $\delta$} & \multicolumn{1}{c}{$\xi$}      & \multicolumn{1}{c}{termica areica}                                            & \multicolumn{1}{c}{termica}                                          & \multicolumn{1}{c}{Termica}                                                                \\
\multicolumn{1}{c}{}                        & \multicolumn{1}{c}{$\left[\SI{}{\metre}\right]$} & \multicolumn{1}{c}{$\left[\SI{}{\kilo\gram\per\metre\cubed}\right]$} & \multicolumn{1}{c}{$\left[\SI{}{\joule\per\kilo\gram\per\kelvin}\right]$} & \multicolumn{1}{c}{$\left[\SI{}{\kilo\gram\per\metre\squared}\right]$} & \multicolumn{1}{c}{$\left[\SI{}{\metre}\right]$}      & \multicolumn{1}{c}{$[-]$} & \multicolumn{1}{c}{$\left[\SI{}{\kilo\joule\per\metre\squared\per\kelvin}\right]$} & \multicolumn{1}{c}{$\left[\SI{.e-7}{\metre\squared\per\second}\right]$} & \multicolumn{1}{c}{$\left[\SI{}{\watt\second\tothe{0.5}\per\metre\squared\per\kelvin}\right]$} \\
\midrule
Gessofibra            & 0,013 & 1150,0 & 1100,0 & 14,95 & 0,068 & 0,192 & 16,4 & 1,66 & 515,4 \\
X-LAM KLH             & 0,096 & 500,0  & 1600,0 & 48,00 & 0,067 & 1,436 & 76,8 & 1,62 & 322,5 \\
Isolante VentirockDuo & 0,105 & 70,0   & 1030,0 & 7,35  & 0,116 & 0,909 & 7,6  & 4,85 & 50,2 \\
Intonaco calce        & 0,015 & 1800,0 & 1000,0 & 27,00 & 0,117 & 0,128 & 27,0 & 5,00 & 1272,8 \\
\bottomrule
\end{tabular}%
}
\end{table}

\begin{flushleft}
\begin{align*}
\text{Massa superficiale totale} M_s \, &= \SI{97.3}{\kilo\gram\per\metre\squared}\\
\text{Sfasamento} \, \Delta\tau &= \SI{8.34}{\hour}\\
\text{Fattore di attenuazione} \, fd &= \SI{0.368}{}\\
\text{Trasmittanza termica periodica} \, Y_{12} &= \SI{0.092}{\watt\per\metre\squared\per\kelvin}\\
\text{Ammettanza termica interna} \, Y_{11} &= \SI{2.617}{\watt\per\metre\squared\per\kelvin}\\
\text{Ammettanza termica esterna} \, Y_{22} &= \SI{2.156}{\watt\per\metre\squared\per\kelvin}\\
\text{Capacità termica periodica interna} \, k_1 &= \SI{37.19}{\kilo\joule\per\metre\squared\per\kelvin}\
\end{align*}
\end{flushleft}
