\begin{table}
\centering
\caption{Parete in laterizio con isolante interno}
\begin{tabular}{lrrrrrrr}
\toprule
                Strati & Spessori & Densità & Calore specifico & Massa superficiale & Profondità di Penetrazione &     xi & Capacità termica & Diffusività termica x 10-7 & Effusività termcia \\
\midrule
              Intonaco &    0,015 &  1500,0 &           1000,0 &               22,5 &                      0,105 &  0,143 &             22,5 &                       4,00 &              948,7 \\
 Isolante VentirockDuo &    0,120 &    70,0 &           1030,0 &                8,4 &                      0,116 &  1,039 &              8,7 &                       4,85 &               50,2 \\
   Laterizio semipieno &    0,200 &  1000,0 &            840,0 &              200,0 &                      0,132 &  1,518 &            168,0 &                       6,31 &              667,2 \\
              Intonaco &    0,015 &  1800,0 &           1000,0 &               27,0 &                      0,117 &  0,128 &             27,0 &                       5,00 &             1272,8 \\
\bottomrule
\end{tabular}
\end{table}

\begin{flushleft}
\begin{align*}
\text{Massa superficiale totale} \, M_s &= \SI{257.9}{\kilo\gram\per\metre\squared}\\
\text{Sfasamento} \, \Delta\tau &= \SI{9.09}{\hour}\\
\text{Fattore di attenuazione} \, fd &= \SI{0.352}{}\\
\text{Trasmittanza termica periodica} \, Y_{12} &= \SI{0.088}{\watt\per\metre\squared\per\kelvin}\\
\text{Ammettanza termica interna} \, Y_{11} &= \SI{1.744}{\watt\per\metre\squared\per\kelvin}\\
\text{Ammettanza termica esterna} \, Y_{22} &= \SI{5.991}{\watt\per\metre\squared\per\kelvin}\\
\text{Capacità termica periodica interna} \, k_1 &= \SI{25.09}{\kilo\joule\per\metre\squared\per\kelvin}\\
\end{align*}
\end{flushleft}
